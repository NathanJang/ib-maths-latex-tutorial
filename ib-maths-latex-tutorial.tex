% This line is a comment.
% Lines beginning with `%' will be ignored by the compiler.

% Here begins the preamble: stuff that (usually) won't be shown in the actual document but is needed to set up the document.

% Declare the document class.
% In this case, it's the custom class that I wrote for use in all my documents.
% Found in `chan.cls' or https://gist.github.com/NathanJang/bf6c5f5080364b7811b5
% Feel free to use in your own projects.
\documentclass{chan}
% Make the compiler load some useful external packages for extra functionality, to modify styling, etc.
\usepackage{float, listings, parskip, url}

% Declare the title.
\title{\LaTeX\ Tutorial}
% Declare the date.
\date{2015-08-20}
% Declare this document's author.
%\author{Jonathan Chan}

% The following declarations are features found only in my document class; ignore for other classes.
% Declare an empty subtitle.
\subtitle{}
% Declare the school course for this document.
\course{IB Mathematics HL}
% Declare the course's instructor.
\instructor{Ms. Huang}

% Declare any other configuration stuff here.
\lstset{language=TeX, basicstyle=\ttfamily}

% Here ends the preamble.

% Here begins the document body that will show up in print.
\begin{document}

% Output the title, generated from the configuration in the preamble.
%\maketitle

% Declare the abstract, a gist of the document.
\abstract{How to write a document in \LaTeX.}

% Make a new section.
\section{Introduction}

\LaTeX\ is a typesetting language that is widely used in academia, especially in Science and Mathematics.
Every \LaTeX\ document is written in a text file, usually with a \texttt{.tex} file extension.
You can then use a ``compiler'' to convert the text file to a PDF that you can print out or share.
Because documents are written in text files, you can write your documents with a focus on content, and \LaTeX\ handles the formatting for you.
It also means that you don't need any special programs to edit document; you can edit on any OS and compile it online if needed.

Every document should be contained in its own folder, along with associated content like images.

\subsection{Compiling}

There is a compiler called ``MacTeX'' for integrated editing and compiling documents on Mac OS X.
For now, we can use \url{https://sharelatex.com} to write and compile documents.

\section{Basics}

A \LaTeX\ command begins with a backslash (``\verb$\$''), followed by arguments surrounded by \verb${}$ immediately after.
Options may optionally be included before the \{ in square brackets (``\verb$[]$'')

\begin{lstlisting}
\somecommand[options]{arguments}
\end{lstlisting}

A \LaTeX\ ``environment'' is like a command, but for longer content.

\begin{lstlisting}
\begin{someenvironment}
content
\end{someenvironment}
\end{lstlisting}

Every command has different usages, so it's important to read the documentation before using one.

\subsection{Document Structure}

Every document must start with a preamble, which includes stuff that configures the document and will not be displayed in the final document.
Here is a sample preamble:

\begin{lstlisting}
\documentclass[a4paper]{article}
\usepackage{apacite}

\author{Justin LoXue}
\date{2015-08-15}
\title{Why Tortoises are Better than Turtles}
\end{lstlisting}

\subsubsection{Document Class}

\lstinline$\documentclass{article}$ tells the compiler to use the \verb$article$ class.
Every document must declare a class, which is a sort of template to use when compiling a document.
The \verb$[a4paper]$ tells the \verb$article$ class to make the PDF with an A4 paper size.

\subsubsection{Packages}

You can load third-party packages with \lstinline$\usepackage{}$.
Packages can be used to add extra functionality or change styling.
\verb$apacite$ can be used to auto-format APA citations.

Every package has different usage.
You should read the documentation or examples online before using a package.

\subsubsection{Metadata}

You can use \lstinline$\author{}$, \lstinline$\date{}$, and \lstinline$\title{}$ to set the metadata for the document.

\subsubsection{Document Environment}

After the preamble, a document's content must be contained in a \verb$document$ environment.
Optionally, you can (usually) add \lstinline$\maketitle$ at the top to generate a title from the preamble's metadata.

\begin{lstlisting}
\begin{document}
\maketitle

...
\end{document}
\end{lstlisting}

To write normal paragraphs, simply type as usual in the \verb$document$ environment.
Use 2 newlines to start a new paragraph (press enter twice).
This usage is recommended:

\begin{lstlisting}
This is a paragraph.
You should start each new sentence on a new line for better organisation.
The compiler puts it all in the same paragraph.

This is a second paragraph.
\end{lstlisting}

Gets you:

This is a paragraph.
You should start each new sentence on a new line for better organisation.
The compiler puts it all in the same paragraph.

This is a second paragraph.

% Here ends the document.
\end{document}

% There must not be anything else after `\end{document}'.
