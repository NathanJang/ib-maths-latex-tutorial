% This line is a comment.
% Lines beginning with `%' will be ignored by the compiler.

% Here begins the preamble: stuff that (usually) won't be shown in the actual document but is needed to set up the document.

% Declare the document class.
% In this case, it's the custom class that I wrote for use in all my documents.
% Found in `chan.cls' or https://gist.github.com/NathanJang/bf6c5f5080364b7811b5
% Feel free to use in your own projects.
\documentclass{chan}
% Make the compiler load some useful external packages for extra functionality, to modify styling, etc.
\usepackage{float, graphicx, listings, longtable, parskip, tikz, upquote, url}

% Declare the title.
\title{\LaTeX\ Tutorial}
% Declare the date.
\date{2015-08-20}
% Declare this document's author.
% Commenting it out because the `chan.cls' class does it for me already, which is why I use it :)
%\author{Jonathan Chan}

% The following declarations are features found only in my document class; ignore for other classes.
% Declare an empty subtitle.
\subtitle{}
% Declare the school course for this document.
\course{IB Mathematics HL}
% Declare the course's instructor.
\instructor{Ms. Huang}

% Declare any other configuration stuff here.
\lstset{language=TeX, basicstyle=\ttfamily}

% Here ends the preamble.

% Here begins the document body that will show up in print.
\begin{document}

% Output the title, generated from the configuration in the preamble.
%\maketitle

% Declare the abstract, a gist of the document.
\abstract{How to write a document in \LaTeX.}

% Make a new section.
\section{Introduction}

\LaTeX\ is a typesetting language that is widely used in academia, especially in Science and Mathematics.
Every \LaTeX\ document is written in a text file, usually with a \verb$.tex$ file extension.
You can then use a ``compiler'' to convert the text file to a PDF that you can print out or share.
Because documents are written in text files, you can write your documents with a focus on content, and \LaTeX\ handles the formatting for you.
It also means that you don't need any special programs to edit document; you can edit on any OS and compile it online if needed.

Every document should be contained in its own folder, along with associated content like images.

If you're interested, this document was made in \LaTeX\ and its source can be found at \url{https://github.com/NathanJang/ib-maths-latex-tutorial}.

\subsection{Compiling}

There is a compiler called ``MacTeX'' for integrated editing and compiling documents on Mac OS X.
For now, we can use \url{https://overleaf.com} to write and compile documents.

\section{Basics}

A \LaTeX\ command begins with a backslash (``\verb$\$''), followed by arguments surrounded by \verb${}$ immediately after.
Options may optionally be included before the \{ in square brackets (``\verb$[]$'')

\begin{lstlisting}
\somecommand[options]{arguments}
\end{lstlisting}

A \LaTeX\ ``environment'' is like a command, but for longer content.

\begin{lstlisting}
\begin{someenvironment}[options]
content
\end{someenvironment}
\end{lstlisting}

Every command has a different usage, so it's important to read the documentation before using one.

If you like, you can add ``comments'' to your \verb$.tex$ file.
Comments are lines that can be saved in the source code of your document but will be ignored by the compiler and will not show up in the PDF.
To do this, add a percent symbol (``\verb$%$'') at the beginning of a comment.
The rest of the line will be hidden.

\subsection{Document Structure}

Every document must start with a preamble, which includes stuff that configures the document and will not be displayed in the final document.
Here is a sample preamble:

\begin{lstlisting}
\documentclass[a4paper]{article}
\usepackage{apacite}

\author{Justin LoXue}
\date{2015-08-15}
\title{Why Tortoises are Better than Turtles}
\end{lstlisting}

\subsubsection{Document Class}

\lstinline$\documentclass{article}$ tells the compiler to use the \verb$article$ class.
Every document must declare a class, which is a sort of template to use when compiling a document.
The \verb$[a4paper]$ tells the \verb$article$ class to make the PDF with an A4 paper size.

\subsubsection{Packages}

You can load third-party packages with \lstinline$\usepackage{}$.
Packages can be used to add extra functionality or change styling.
\verb$apacite$ can be used to auto-format APA citations.

Every package has different usage.
You should read the documentation or examples online before using a package.

\subsubsection{Metadata}

You can use \lstinline$\author{}$, \lstinline$\date{}$, and \lstinline$\title{}$ to set the metadata for the document.

\subsubsection{Document Environment}

After the preamble, a document's content must be contained in a \verb$document$ environment.
Optionally, you can (usually) add \lstinline$\maketitle$ at the top to generate a title from the preamble's metadata.

\begin{lstlisting}
\begin{document}
\maketitle

...
\end{document}
\end{lstlisting}

To write normal paragraphs, simply type as usual in the \verb$document$ environment.
Use 2 newlines to start a new paragraph (press enter twice).
This usage is recommended:

\begin{lstlisting}
This is a paragraph.
You should put each sentence on a separate line for better organisation.
The compiler puts it all in the same paragraph.

This is a second paragraph.
\end{lstlisting}

Gets you:

This is a paragraph.
You should put each sentence on a separate line for better organisation.
The compiler puts it all in the same paragraph.

This is a second paragraph.

\section{Useful Functionality}

For quotations \emph{don't} use the usual \verb$""$.
Instead, use backtick(s) and apostrophe(s) for \verb$`single quotes'$ and \verb$``double quotes''$.

\subsection{Typography}

\begin{longtable}{c|c|c}
type & mnemonic & command\\
\hline
\endhead
\textbf{bold} & text boldface (bf) & \lstinline$\textbf{}$\\
\emph{italics} & emphasis & \lstinline$\emph{}$ or \lstinline$\textit{}$\\
\underline{underline} & underline & \lstinline$\underline{}$\\
\caption{Styling commands.}
\label{tab:styling}
\end{longtable}

It's suggested that you use \lstinline$\emph{}$ (italics) over the others because of readability, hence its name (\emph{emphasis}).

\subsubsection{Fonts}

There are 3 types of fonts here: \textrm{serif}, \textsf{sans-serif}, and \texttt{monospace}.
You can use these fonts inline or apply them to the whole document.

\begin{longtable}{c|c|c|c}
type & abbreviation & inline & whole document\\
\hline
\textrm{serif} & rm (Roman) & \lstinline$\textrm{}$ & \lstinline$\rmfamily$\\
\textsf{sans-serif} & sf (sans-serif) & \lstinline$\textsf{}$ & \lstinline$\sffamily$\\
\texttt{monospace} & tt (typewriter) & \lstinline$\texttt{}$ & \lstinline$\ttfamily$\\
\caption{Different font family commands.}
\label{tab:fonttable}
\end{longtable}

If you're required to use Times New Roman, unfortunately it isn't available.
However, you can use Times which is close enough, by adding \lstinline$\usepackage{times}$ to your preamble.
\LaTeX's default font is allegedly optimised for readability.

\subsection{Lists}

Use either the \verb$itemize$ or \verb$enumerate$ environment, and declare each point with \lstinline$\item$.
You can nest them if you want.

\begin{lstlisting}
\begin{enumerate}
\item A robot may not injure a human being or, through inaction,
    allow a human being to come to harm.
    \begin{itemize}
    \item A similar 0th Law was later added:
        A robot may not harm humanity, or, by inaction,
        allow humanity to come to harm.
    \end{itemize}
\item A robot must obey the orders given it by human beings,
    except when it conflicts with the first law.
\item A robot must protect its own existence as long as
    such protection does not conflict with the First or Second Laws.
\end{enumerate}
\end{lstlisting}

Gets you:

\begin{enumerate}
\item A robot may not injure a human being or, through inaction,
    allow a human being to come to harm.
    \begin{itemize}
    \item A similar 0th Law was later added:
        A robot may not harm humanity, or, by inaction,
        allow humanity to come to harm.
    \end{itemize}
\item A robot must obey the orders given it by human beings,
    except when it conflicts with the first law.
\item A robot must protect its own existence as long as
    such protection does not conflict with the First or Second Laws.
\end{enumerate}

\subsection{Floats}

``Floats'' are blocks of content that the compiler keeps together in one chunk on one page.
These can be used on equations, tables, images, etc.
They all use similar commands.

\subsection{Mathematics}

Mathematics mode is a different environment than the usual text mode where you type paragraphs and text.
It provides special formatting and extra commands that cannot be used in text mode.
\emph{Always remember to put maths in a maths environment.}

By default, any letters you put in a maths environment will be italicised because the compiler thinks they're variables multiplied together.

\lstinline$\[ \frac{time spent formatting document}{time spent on essay} > 1 \]$

Gets you:

\[ \frac{time spent formatting document}{time spent on essay} > 1 \]

This is no good.

You should put them in a \lstinline$\textrm{}$.

\lstinline$\[ \frac{\textrm{time spent formatting document}}{\textrm{time spent on essay}} > 1 \]$

Gets you:

\[ \frac{\textrm{time spent formatting document}}{\textrm{time spent on essay}} > 1 \]

\subsubsection{Inline}

To add mathematics inline, put them in \lstinline$\(  \)$.

\begin{lstlisting}
What's the difference between you and I? \( u - i \).
\end{lstlisting}

Gets you:

What's the difference between you and I? \( u - i \).

\subsubsection{Block}

In a block, maths will be centred and take up some vertical space.

For quick usage put them in \lstinline$\[  \]$.

\begin{lstlisting}
\[ n = oi + ce \]
\end{lstlisting}

Gets you:

\[ n = oi + ce \]

For enhanced usage, put them in the \verb$equation$ environment.

\begin{lstlisting}
Equation \ref{eq:survival} almost describes how humanity is able to survive.

\begin{equation}
\lim_{i\to\infty} (\sqrt[n]{e^x} - \frac{1}{\sqrt[n]{e^i}})
= \frac{d}{dx} f(u)
\label{eq:survival}
\end{equation}
\end{lstlisting}

Gets you:

Equation \ref{eq:survival} almost describes how humanity is able to survive.

\begin{equation}
\lim_{i\to\infty} (\sqrt[n]{e^x} - \frac{1}{\sqrt[n]{e^i}})
= \frac{d}{dx} f(u)
\label{eq:survival}
\end{equation}

\subsubsection{Common Maths Commands}

\begin{longtable}{c|c}
symbol & command\\
\hline
\endhead
\(\frac{a}{b}\) & \lstinline$\frac{a}{b}$\\
\(\times\) & \lstinline$\times$\\
\(\sqrt{x}\) & \lstinline$\sqrt{x}$\\
\(\sqrt[n]{x}\) & \lstinline$\sqrt[n]{x}$\\
\(\pi\) & \lstinline$\pi$\\
\(x^n\) & \lstinline$x^n$\\
\(x_1\) & \lstinline$x_1$\\
\(\Sigma_{i=0}^n\) & \lstinline$\Sigma_{i=0}^n$\\
\(\int_{i=0}^n\) & \lstinline$\int_{i=0}^n$\\
\(\infty\) & \lstinline$\infty$\\
\(\approx\) & \lstinline$\approx$\\
\(\to\) & \lstinline$\to$\\
\(f'(x)\) & \lstinline$f'(x)$\\
\(f''(x)\) & \lstinline$f''(x)$\\
\caption{Common maths commands.}
\label{tab:mathscmd}
\end{longtable}

\section{Tables}

\begin{lstlisting}
Table \ref{tab:0118999} outlines the new emergency services number.

\begin{table}[h]
\centering
\begin{tabular}{r|c|l}
right justified & centred & left justified\\
\hline
oh & one & one\\
eight & nine & nine\\
nine & eight & eight\\
one & nine & nine\\
nine & one & one\\
nine & seven & two\\
five & three & oh\\
\end{tabular}
\caption{0118999\ldots}
\label{tab:0118999}
\end{table}
\end{lstlisting}

Gets you:

Table \ref{tab:0118999} outlines the new emergency services number.

\begin{table}[h]
\centering
\begin{tabular}{r|c|l}
right justified & centred & left justified\\
\hline
oh & one & one\\
eight & nine & nine\\
nine & eight & eight\\
one & nine & nine\\
nine & one & one\\
nine & seven & two\\
five & three & oh\\
\end{tabular}
\caption{0118999\ldots}
\label{tab:0118999}
\end{table}

\lstinline$\begin{table}[h]$ starts a table object (wrapper).
\verb$h$ means approximately here.
You can also use \verb$t$, \verb$b$, or \verb$H$ for top, bottom, or Here exactly, respectively.
\lstinline$\centering$ means to centre the table on the page.

\lstinline$\begin{tabular}{r|c|l}$ begins the actual table with content;
\verb${r|c|l}$ means to align 3 columns right, centred, and left respectively.
This must match the number of columns.
(You can omit the bar (``\verb$|$'') to omit vertical lines on the table.)

\lstinline$\hline$ draws a horizontal line (you can draw these on any row).

\lstinline$oh & one & one$ is a row, with the ampersand (``\verb$&$'') separating columns.
\lstinline$\\$ means a new row.

At the end of the table, \lstinline$\caption{}$ puts a caption under the table.
\lstinline$\label{}$ adds a reference identifier to this table that you can refer to with \lstinline$\ref{}$ outside the table.
It's conventional to prefix \verb$tab:$ to table identifiers.
\emph{Label must be after caption because of how the compiler counts stuff.}

\section{Figures}

Figures have the general form:

\begin{lstlisting}
\begin{figure}[h]
...
\caption{Caption text.}
\label{fig:someidentifier}
\end{figure}
\end{lstlisting}

similar to the \verb$table$ environment.

\subsection{Images}

\lstinline$\usepackage{graphicx}$ to include images.

\begin{lstlisting}
\begin{figure}[h]
\centering
\includegraphics[width=5cm]{deddit}
\caption{It is indeed.}
\label{fig:deddit}
\end{figure}
\end{lstlisting}

Gets you:

\begin{figure}[h]
\centering
\includegraphics[width=5cm]{deddit}
\caption{It is indeed.}
\label{fig:deddit}
\end{figure}

Remember to add \lstinline$\centering$ if you want.

See \url{https://en.wikibooks.org/wiki/LaTeX/Importing_Graphics#Including_graphics} for additional options.

\subsection{Drawing Geometrical Shapes}

\lstinline$\usepackage{tikz}$ if you want prettier graphs than screenshots of Excel, but you can draw any lines with this.

\subsubsection{Basics}

\begin{figure}[h]
\centering
\begin{tikzpicture}
\draw (-1,3) -- (-1,0);
\draw (1,3) -- (1,0);
\draw (-2,0) -- (0,-2) -- (2,0);
\end{tikzpicture}
\end{figure}

In \verb$tikz$, every command ends with a semicolon (``;'').
The most basic command is \lstinline$\draw$, which draws straight lines defined by cartesian points.

\lstinline$\draw (0,0) -- (1,1) -- (2,0);$ draws straight lines to connect all these points.

\lstinline$\node at (0,0) {thing};$ puts the text ``thing'' at (0,0).

There are more complicated commands for polygons, curves, etc., but for our purposes this is enough.

\subsubsection{Graphs}

\begin{lstlisting}
\begin{figure}[H]
\centering
\begin{tikzpicture}
\draw [->] (0,0) -- (13,0) node [anchor=west] {\(t\)};
\foreach \x in {2,4,...,20}
    \draw (\x*0.6,1pt) -- (\x*0.6,-1pt) node [anchor=north] {\(\x\)};
\draw [<->] (0,-5) -- (0,5) node [anchor=south] {\(y(t)\)};
\foreach \y in {-100,-80,...,100}
    \draw (1pt,\y/21) -- (-1pt,\y/21) node [anchor=east] {\(\y\)};

\draw [->] (0.000,0.000) -- (0.060,-0.492) -- (0.120,-0.966) -- 
(0.180,-1.417) -- (0.240,-1.842) -- (0.300,-2.237) -- (0.360,-2.597) -- 
(0.420,-2.919) -- (0.480,-3.202) -- (0.540,-3.443) -- (0.600,-3.639) -- 
(0.660,-3.791) -- (0.720,-3.898) -- (0.780,-3.959) -- (0.840,-3.976) -- 
(0.900,-3.948) -- (0.960,-3.877) -- (1.020,-3.766) -- (1.080,-3.616) -- 
(1.140,-3.431) -- (1.200,-3.212) -- (1.260,-2.963) -- (1.320,-2.688) -- 
(1.380,-2.390) -- (1.440,-2.074) -- (1.500,-1.742) -- (1.560,-1.399) -- 
(1.620,-1.050) -- (1.680,-0.698) -- (1.740,-0.346) -- (1.800,-0.000) -- 
(1.860,0.338) -- (1.920,0.664) -- (1.980,0.974) -- (2.040,1.266) -- 
(2.100,1.537) -- (2.160,1.785) -- (2.220,2.006) -- (2.280,2.201) -- 
(2.340,2.366) -- (2.400,2.501) -- (2.460,2.606) -- (2.520,2.679) -- 
(2.580,2.721) -- (2.640,2.732) -- (2.700,2.713) -- (2.760,2.665) -- 
(2.820,2.588) -- (2.880,2.485) -- (2.940,2.358) -- (3.000,2.207) -- 
(3.060,2.036) -- (3.120,1.847) -- (3.180,1.643) -- (3.240,1.425) -- 
(3.300,1.197) -- (3.360,0.962) -- (3.420,0.722) -- (3.480,0.480) -- 
(3.540,0.238) -- (3.600,0.000) -- (3.660,-0.232) -- (3.720,-0.456) -- 
(3.780,-0.670) -- (3.840,-0.870) -- (3.900,-1.057) -- (3.960,-1.227) -- 
(4.020,-1.379) -- (4.080,-1.513) -- (4.140,-1.626) -- (4.200,-1.719) -- 
(4.260,-1.791) -- (4.320,-1.841) -- (4.380,-1.870) -- (4.440,-1.878) -- 
(4.500,-1.865) -- (4.560,-1.832) -- (4.620,-1.779) -- (4.680,-1.708) -- 
(4.740,-1.620) -- (4.800,-1.517) -- (4.860,-1.400) -- (4.920,-1.270) -- 
(4.980,-1.129) -- (5.040,-0.979) -- (5.100,-0.823) -- (5.160,-0.661) -- 
(5.220,-0.496) -- (5.280,-0.330) -- (5.340,-0.164) -- (5.400,-0.000) -- 
(5.460,0.160) -- (5.520,0.313) -- (5.580,0.460) -- (5.640,0.598) -- 
(5.700,0.726) -- (5.760,0.843) -- (5.820,0.948) -- (5.880,1.040) -- 
(5.940,1.118) -- (6.000,1.182) -- (6.060,1.231) -- (6.120,1.265) -- 
(6.180,1.285) -- (6.240,1.291) -- (6.300,1.282) -- (6.360,1.259) -- 
(6.420,1.223) -- (6.480,1.174) -- (6.540,1.114) -- (6.600,1.043) -- 
(6.660,0.962) -- (6.720,0.873) -- (6.780,0.776) -- (6.840,0.673) -- 
(6.900,0.566) -- (6.960,0.454) -- (7.020,0.341) -- (7.080,0.227) -- 
(7.140,0.112) -- (7.200,0.000) -- (7.260,-0.110) -- (7.320,-0.215) -- 
(7.380,-0.316) -- (7.440,-0.411) -- (7.500,-0.499) -- (7.560,-0.579) -- 
(7.620,-0.651) -- (7.680,-0.714) -- (7.740,-0.768) -- (7.800,-0.812) -- 
(7.860,-0.846) -- (7.920,-0.870) -- (7.980,-0.883) -- (8.040,-0.887) -- 
(8.100,-0.881) -- (8.160,-0.865) -- (8.220,-0.840) -- (8.280,-0.807) -- 
(8.340,-0.765) -- (8.400,-0.717) -- (8.460,-0.661) -- (8.520,-0.600) -- 
(8.580,-0.533) -- (8.640,-0.463) -- (8.700,-0.389) -- (8.760,-0.312) -- 
(8.820,-0.234) -- (8.880,-0.156) -- (8.940,-0.077) -- (9.000,-0.000) -- 
(9.060,0.075) -- (9.120,0.148) -- (9.180,0.217) -- (9.240,0.283) -- 
(9.300,0.343) -- (9.360,0.398) -- (9.420,0.448) -- (9.480,0.491) -- 
(9.540,0.528) -- (9.600,0.558) -- (9.660,0.581) -- (9.720,0.598) -- 
(9.780,0.607) -- (9.840,0.610) -- (9.900,0.605) -- (9.960,0.595) -- 
(10.02,0.578) -- (10.08,0.555) -- (10.14,0.526) -- (10.20,0.493) -- 
(10.26,0.454) -- (10.32,0.412) -- (10.38,0.367) -- (10.44,0.318) -- 
(10.50,0.267) -- (10.56,0.215) -- (10.62,0.161) -- (10.68,0.107) -- 
(10.74,0.053) -- (10.80,0.000) -- (10.86,-0.052) -- (10.92,-0.102) -- 
(10.98,-0.149) -- (11.04,-0.194) -- (11.10,-0.236) -- (11.16,-0.274) -- 
(11.22,-0.308) -- (11.28,-0.337) -- (11.34,-0.363) -- (11.40,-0.384) -- 
(11.46,-0.400) -- (11.52,-0.411) -- (11.58,-0.417) -- (11.64,-0.419) -- 
(11.70,-0.416) -- (11.76,-0.409) -- (11.82,-0.397) -- (11.88,-0.381) -- 
(11.94,-0.362) -- (12.00,-0.339);
\end{tikzpicture}
\caption{A plot of the bungee jumping model.}
\label{fig:bungeeReal}
\end{figure}
\end{lstlisting}

Gets you:

\begin{figure}[H]
\centering
\begin{tikzpicture}
\draw [->] (0,0) -- (13,0) node [anchor=west] {\(t\)};
\foreach \x in {2,4,...,20}
    \draw (\x*0.6,1pt) -- (\x*0.6,-1pt) node [anchor=north] {\(\x\)};
\draw [<->] (0,-5) -- (0,5) node [anchor=south] {\(y(t)\)};
\foreach \y in {-100,-80,...,100}
    \draw (1pt,\y/21) -- (-1pt,\y/21) node [anchor=east] {\(\y\)};

\draw [->] (0.000,0.000) -- (0.060,-0.492) -- (0.120,-0.966) -- 
(0.180,-1.417) -- (0.240,-1.842) -- (0.300,-2.237) -- (0.360,-2.597) -- 
(0.420,-2.919) -- (0.480,-3.202) -- (0.540,-3.443) -- (0.600,-3.639) -- 
(0.660,-3.791) -- (0.720,-3.898) -- (0.780,-3.959) -- (0.840,-3.976) -- 
(0.900,-3.948) -- (0.960,-3.877) -- (1.020,-3.766) -- (1.080,-3.616) -- 
(1.140,-3.431) -- (1.200,-3.212) -- (1.260,-2.963) -- (1.320,-2.688) -- 
(1.380,-2.390) -- (1.440,-2.074) -- (1.500,-1.742) -- (1.560,-1.399) -- 
(1.620,-1.050) -- (1.680,-0.698) -- (1.740,-0.346) -- (1.800,-0.000) -- 
(1.860,0.338) -- (1.920,0.664) -- (1.980,0.974) -- (2.040,1.266) -- 
(2.100,1.537) -- (2.160,1.785) -- (2.220,2.006) -- (2.280,2.201) -- 
(2.340,2.366) -- (2.400,2.501) -- (2.460,2.606) -- (2.520,2.679) -- 
(2.580,2.721) -- (2.640,2.732) -- (2.700,2.713) -- (2.760,2.665) -- 
(2.820,2.588) -- (2.880,2.485) -- (2.940,2.358) -- (3.000,2.207) -- 
(3.060,2.036) -- (3.120,1.847) -- (3.180,1.643) -- (3.240,1.425) -- 
(3.300,1.197) -- (3.360,0.962) -- (3.420,0.722) -- (3.480,0.480) -- 
(3.540,0.238) -- (3.600,0.000) -- (3.660,-0.232) -- (3.720,-0.456) -- 
(3.780,-0.670) -- (3.840,-0.870) -- (3.900,-1.057) -- (3.960,-1.227) -- 
(4.020,-1.379) -- (4.080,-1.513) -- (4.140,-1.626) -- (4.200,-1.719) -- 
(4.260,-1.791) -- (4.320,-1.841) -- (4.380,-1.870) -- (4.440,-1.878) -- 
(4.500,-1.865) -- (4.560,-1.832) -- (4.620,-1.779) -- (4.680,-1.708) -- 
(4.740,-1.620) -- (4.800,-1.517) -- (4.860,-1.400) -- (4.920,-1.270) -- 
(4.980,-1.129) -- (5.040,-0.979) -- (5.100,-0.823) -- (5.160,-0.661) -- 
(5.220,-0.496) -- (5.280,-0.330) -- (5.340,-0.164) -- (5.400,-0.000) -- 
(5.460,0.160) -- (5.520,0.313) -- (5.580,0.460) -- (5.640,0.598) -- 
(5.700,0.726) -- (5.760,0.843) -- (5.820,0.948) -- (5.880,1.040) -- 
(5.940,1.118) -- (6.000,1.182) -- (6.060,1.231) -- (6.120,1.265) -- 
(6.180,1.285) -- (6.240,1.291) -- (6.300,1.282) -- (6.360,1.259) -- 
(6.420,1.223) -- (6.480,1.174) -- (6.540,1.114) -- (6.600,1.043) -- 
(6.660,0.962) -- (6.720,0.873) -- (6.780,0.776) -- (6.840,0.673) -- 
(6.900,0.566) -- (6.960,0.454) -- (7.020,0.341) -- (7.080,0.227) -- 
(7.140,0.112) -- (7.200,0.000) -- (7.260,-0.110) -- (7.320,-0.215) -- 
(7.380,-0.316) -- (7.440,-0.411) -- (7.500,-0.499) -- (7.560,-0.579) -- 
(7.620,-0.651) -- (7.680,-0.714) -- (7.740,-0.768) -- (7.800,-0.812) -- 
(7.860,-0.846) -- (7.920,-0.870) -- (7.980,-0.883) -- (8.040,-0.887) -- 
(8.100,-0.881) -- (8.160,-0.865) -- (8.220,-0.840) -- (8.280,-0.807) -- 
(8.340,-0.765) -- (8.400,-0.717) -- (8.460,-0.661) -- (8.520,-0.600) -- 
(8.580,-0.533) -- (8.640,-0.463) -- (8.700,-0.389) -- (8.760,-0.312) -- 
(8.820,-0.234) -- (8.880,-0.156) -- (8.940,-0.077) -- (9.000,-0.000) -- 
(9.060,0.075) -- (9.120,0.148) -- (9.180,0.217) -- (9.240,0.283) -- 
(9.300,0.343) -- (9.360,0.398) -- (9.420,0.448) -- (9.480,0.491) -- 
(9.540,0.528) -- (9.600,0.558) -- (9.660,0.581) -- (9.720,0.598) -- 
(9.780,0.607) -- (9.840,0.610) -- (9.900,0.605) -- (9.960,0.595) -- 
(10.02,0.578) -- (10.08,0.555) -- (10.14,0.526) -- (10.20,0.493) -- 
(10.26,0.454) -- (10.32,0.412) -- (10.38,0.367) -- (10.44,0.318) -- 
(10.50,0.267) -- (10.56,0.215) -- (10.62,0.161) -- (10.68,0.107) -- 
(10.74,0.053) -- (10.80,0.000) -- (10.86,-0.052) -- (10.92,-0.102) -- 
(10.98,-0.149) -- (11.04,-0.194) -- (11.10,-0.236) -- (11.16,-0.274) -- 
(11.22,-0.308) -- (11.28,-0.337) -- (11.34,-0.363) -- (11.40,-0.384) -- 
(11.46,-0.400) -- (11.52,-0.411) -- (11.58,-0.417) -- (11.64,-0.419) -- 
(11.70,-0.416) -- (11.76,-0.409) -- (11.82,-0.397) -- (11.88,-0.381) -- 
(11.94,-0.362) -- (12.00,-0.339);
\end{tikzpicture}
\caption{A plot of the bungee jumping model.}
\label{fig:bungeeReal}
\end{figure}

\section{Escaping}

If you ever need to type a reserved character into the actual document, try adding a backslash (``\verb$\$'') before it. For example, to print \verb$%$, type \lstinline$\%$.

(There are other methods.)

\section{Resources}

\begin{itemize}
\item \url{https://en.wikibooks.org/wiki/LaTeX}
\item \url{http://tex.stackexchange.com}
\end{itemize}

% Here ends the document.
\end{document}

% There must not be anything else after `\end{document}'.
